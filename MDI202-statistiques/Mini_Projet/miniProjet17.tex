
\documentclass[11pt,a4paper,oneside]{article}
\usepackage{TDheader}
\usepackage{enumitem}
\lhead{MDI 220}
\chead{Statistiques}
\rhead{année 2017/2018}
\cfoot{--- \thepage\ ---}
\pagestyle{fancy}

%%%%%%%%%%%%%%%%%%%%%%%%%%%%%%%%%%%%%%%%%%%%%%%%%%%%%%%%%%%%%%%%%%%%%%%%%%
% general-def.tex       Definitions communes a tout le texte
%
%%%%%%%%%%%%%%%%%%%%%%%%%%%%%%%%%%%%%%%%%%%%%%%%%%%%%%%%%%%%%%%%%%%%%%%%%
%%%%%%%%%%% QUELQUES DEFINITIONS
\newcommand{\addots}
{
    \mkenrn2mu\raisebox{1pt}{.}
    \mkern2mu\raisebox{4pt}{.}
    \mkern2mu\raisebox{7pt}{.}
}
\def\rset{\mathbb{R}}
\def\cset{\mathbb{C}}
\def\zset{\mathbb{Z}}
\def\nset{\mathbb{N}}
\def\qset{\mathbb{Q}}
\def\Cset{\mathbb{C}}
\def\Zset{\mathbb{Z}}
% et d'autres
\newcommand{\signe}{\mathrm{sgn}}
\newcommand{\rect}{\mathrm{rect}}
\newcommand{\sinc}{\mathrm{sinc}}
\newcommand{\var}{\mathrm{var}}
\newcommand{\cov}{\mathrm{cov}}
\newcommand{\vp}{{\rm vp}}
\newcommand{\erf}{{\rm erf}}
\def\mod{{\ \rm mod\ }}
% Environnements
% Environnement encadre pour les modeles
\newenvironment{encadr}%
 {\begin{Sbox}\begin{minipage}}%
 {\end{minipage}\end{Sbox}\fbox{\TheSbox}}
% Listes pour les hypotheses dans les theoremes
\newenvironment{mathlist}%
  {\renewcommand{\labelenumi}{\textnormal{(\arabic{enumi})}}\begin{enumerate}}%
  {\end{enumerate}}
\newcommand{\ind}[2]{ \index{#1 \subitem{#2}} }

% Lettres r�serv�es � des nombres particuliers
\def\rmi{\mathrm{i}}
\def\rme{\mathrm{e}}
\def\rmd{\mathrm{d}}
% Abr�viations (attention : ne doivent pas figurer dans les maths)
\newcommand{\etal}{{\em et al.}}
\newcommand{\ie}{{\em i.e.}}
\newcommand{\NB}{\emph{N.B}}
\newcommand{\ssi}{si et seulement si}
% Non utilisees pour l'instant
\newcommand{\cad}{\hbox{c.-�-d.}}
\newcommand{\cf}{cf.}
\newcommand{\etc}{etc.}
\newcommand{\pex}{p.ex.}
\newcommand{\ps}{p.s.}
\newcommand{\va}{v.a.}
\newcommand{\ea}{e.a.}
\newcommand{\iid}{i.i.d.}
\newcommand{\iidsim}{\overset{\mathrm{i.i.d.}}{\sim}}
\newcommand{\resp}{resp.}
% Alphabet blackboard
\newcommand{\Un}{{\mathrm{1 \mkern-4.2mu I}}}
\newcommand{\eqdef}{\ensuremath{\stackrel{\mathrm{def}}{=}}}
\newcommand{\eqd}{\overset{\text{loi}}{=}}
\newcommand{\ud}{\mathrm{d}} 
\newcommand{\point}{\,\cdot\,}
\newcommand{\bpoint}{\,\boldsymbol{\cdot}\,}
% Probabilites, moments convergence, lois
\def\prob{\mathbb{P}}
\def\esp{\mathbb{E}}
\def\PP{\prob}
\def\PE{\esp}
\def\EE{\PE}
\newcommand{\hEs}{(\Omega,{\cal A}, \prob)}
\newcommand{\Pth}{{\loi_{\theta}}}
\newcommand{\cond}{\vert}
\newcommand{\Ptha}{\loi_{\theta \cond A_{i}}}
\newcommand{\Eth}{{\esp_{\theta}}}
\newcommand{\pth}{{p_{\theta}}}

\def\Var{\mathbb{ V}\mathrm{ar}}
\def\Cov{\mathbb{C}\mathrm{ov}}
\def\Covth{\Cov_{\theta}}
\def\Varth{\Var_{\theta}}

% symboles de loi
\def\loi{\mathrm{P}}
\def\loiemp{\mathrm{P_n}}
\def\loiunif{\mathcal{U}\mathit{nif}}
\def\loibeta{\mathcal{B}\mathit{eta}}
\def\loimulti{\mathcal{M}\mathit{ulti}}
\def\loiber{\mathcal{B}\mathit{er}}
\def\loibin{\mathcal{B}\mathit{in}}
\def\loihyper{\mathcal{H}\mathit{yper}}
\def\loigauss{\mathcal{N}}
\def\loigamma{\mathcal{G}\mathit{amma}}
\def\loipoisson{\mathcal{P}\mathit{oiss}}
\def\loistudent{\mathbf{T}}
\def\loifisher{\mathbf{F}}
\def\loikhi2{\mathbf{\chi^2}}
\def\loiexp{\mathcal{E}}

% Convergence en loi
\newcommand{\cl}{{\:\stackrel{\calL}{\longrightarrow}\:}}
\newcommand{\cla}[1]{{\:\stackrel{\calL_{#1}}{\longrightarrow}\:}}
\newcommand{\clt}{\cla{\theta}}
% Convergence en probabilit�
\newcommand{\cp}{{\:\stackrel{\prob}{\longrightarrow}\:}}
\newcommand{\cpa}[1]{{\:\stackrel{\prob_{#1}}{\longrightarrow}\:}}
\newcommand{\cpt}{\cpa{\theta}}
\newcommand{\cps}{{\:\stackrel{\prob-\mathrm{p.s.}}{\longrightarrow}\:}}
\newcommand{\cpsa}[1]{{\:\stackrel{\prob_{#1}-\mathrm{p.s.}}{\longrightarrow}\:}}
% Identite des distributions
\newcommand{\dist}{\sim}
% Convergence �troite
\newcommand{\ce}{\leadsto}
% Asymptotiquement Gaussien
\newcommand{\isan}[2]{\ \equiv \ {\cal AN}\left (#1,#2 \right )\ }
% Experience statistique
\newcommand{\expe}{(\Xset, \, \mathcal{B}(\Xset),\, (\Pth \,;\, \theta \in \Theta \subset \rset^k))}
\newcommand{\rexpe}{(\Xset, \, \mathcal{B}(\Xset),\, (\Pth \,;\, \theta \in \Theta \subset \rset))}
% Divers
\newcommand{\Tr}{{\mathrm{Tr}}}
\newcommand{\diag}{{\mathrm{diag}}}
\def\vol{{\rm vol}}
\newcommand{\fgauss}{\frac{1}{( 2 \pi \sigma^2 )^{^{\frac{n}{2}}}}}
\newcommand{\range}{{\rm Im}}
\newcommand{\rang}{{\rm rang}}
\def\compose{\circ}
% Dans moments.tex
\newcommand{\ketchit}{{o_{P_\theta}}}
\def\mutu{\mu_1^{(t)}}
\def\mutd{\mu_2^{(t)}}
\def\mt{m^{(t)}}
\def\hmt{\hat{m}_n^{(t)}}
\def\gt{g^{(t)}}
\def\htg{\hat{\theta}_n^\Gamma}
% Adherence
\def\adh{{\rm adh}}
% Differentiation
\newcommand{\Df}[3]{{\nabla_{#1}#2(#3)}}
\newcommand{\DDf}[3]{{\nabla^2_{#1}#2(#3)}}
\newcommand{\CDf}[4]{ {\frac{ \partial #3}{\partial #1_{#2}}(#4)}}
\newcommand{\CDfs}[3]{{\frac{ \partial #2}{\partial #1}(#3)}}
%%%%%%%%%%%%%%%%%%%%%%%%%%%
%% qq lettres en gras et/ou avec des chapeaux
\newcommand{\bI}{{\mathbf I}}
\newcommand{\bzero}{{\mathbf 0}}
\newcommand{\bun}{{\mathbf 1}}
\newcommand{\1}{\mathbbm{1}}
\newcommand{\mb}{\mathbf}
%
% lettres en GRAS
\def\bA{\mathbf{A}}
\def\ba{\mathbf{a}}
\def\bb{\mathbf{b}}
\def\bB{\mathbf{B}}
\def\bC{\mathbf{C}}
\def\bc{\mathbf{c}}
\def\bD{\mathbf{D}}
\def\bd{\mathbf{d}}
\def\bE{\mathbf{E}}
\def\be{\mathbf{e}}
\def\grasf{\mathbf{f}}
\def\bF{\mathbf{F}}
\def\bg{\mathbf{g}}
\def\bG{\mathbf{G}}
\def\bh{\mathbf{h}}
\def\bH{\mathbf{H}}
\def\bJ{\mathbf{J}}
\def\bk{\mathbf{k}}
\def\bK{\mathbf{K}}
\def\bL{\mathbf{L}}
\def\bM{\mathbf{M}}
\def\bn{\mathbf{n}}
\def\bO{\mathbf{O}}
\def\bp{\mathbf{p}}
\def\bP{\mathbf{P}}
\def\bQ{\mathbf{Q}}
\def\bR{\mathbf{R}}
\def\br{\mathbf{r}}
\def\bs{\mathbf{s}}
\def\bS{\mathbf{S}}
\def\bt{\mathbf{t}}
\def\bT{\mathbf{T}}
\def\bu{\mathbf{u}}
\def\bU{\mathbf{U}}
\def\bN{\mathbf{N}}
\def\bv{\mathbf{v}}
\def\bV{\mathbf{V}}
\def\bw{\mathbf{w}}
\def\bW{\mathbf{W}}
\def\bx{\mathbf{x}}
\def\bX{\mathbf{X}}
\def\by{\mathbf{y}}
\def\bY{\mathbf{Y}}
\def\bz{\mathbf{z}}
\def\bZ{\mathbf{Z}}
%
\def\eqsp{\;}

%
% lettres grecques en gras
\def\bGamma{\boldsymbol{\Gamma}}
\def\balpha{\boldsymbol{\alpha}}
\def\bbeta{\boldsymbol{\beta}}
\def\bepsilon{\boldsymbol{\epsilon}}
\def\bphi{\boldsymbol{\phi}}
\def\bnu{\boldsymbol{\nu}}
\def\bmu{\boldsymbol{\mu}}
\def\boeta{\boldsymbol{\eta}}
\def\bgamma{\boldsymbol{\gamma}}
\def\btheta{\boldsymbol{\theta}}
\newcommand{\mbf}[1]{\mbox{\boldmath$#1$}}
% lettres "CHAPEAU"
\newcommand\ha{{\hat a}}
\newcommand{\TnG}{{{\theta}_n^{G}}}
\newcommand\hf{{\hat f}}
\newcommand{\hg}{\hat{g}}
\newcommand{\hr}{{\hat r}}
\newcommand{\hR}{{\hat R}}
\newcommand\hs{{\hat s}}
\newcommand{\hS}{{\hat S}}
\newcommand{\hT}{{\hat T}}
%\newcommand{\hsigmad}{{{\hat \sigma}^2}}
\newcommand{\hbS}{{\mathbf {\hat S}}}
\newcommand{\hbx}{{\mathbf {\hat x}}}
\def\halpha{{\hat \alpha}}
\def\hbeta{{\hat \beta}}
\def\hph{{\hat{\phi}}}
\def\hbph{{\hat{\bphi}}}
\def\hth{{\hat{\theta}}}
\def\hbth{{\hat{\btheta}}}
\def\thj{{\theta^{(j)}}}
\def\thju{{\theta^{(j+1)}}}
% lettres tilde
\newcommand{\tg}{\tilde{g}}
\newcommand{\tth}{{\tilde \theta}}
% Lettres caligraphiques
\def\calL{\mathcal{L}}
\def\calA{\mathcal{A}}
\def\calM{\mathcal{M}}
\def\calI{\mathcal{I}}
\def\calC{\mathcal{C}}
\def\calF{\mathcal{F}}
\def\calS{\mathcal{S}}
\def\calE{\mathcal{E}}
\def\calU{\mathcal{U}}
\def\calX{\mathcal{X}}
\def\calY{\mathcal{Y}}
\def\calZ{\mathcal{Z}}
\def\calP{\mathcal{P}}
\def\calB{\mathcal{B}}
\def\calK{\mathcal{K}}
\def\calR{\mathcal{R}}
\def\calT{\mathcal{T}}
\def\calG{\mathcal{G}}
\def\calM{\mathcal{M}}
\def\cF{\mathcal{F}}
\def\cE{\mathcal{E}}
\def\cB{\mathcal{B}}
\def\limn{\lim_{n \rightarrow \infty}}
\def\cH{\mathcal{H}}
\def\cG{\mathcal{G}}
\def\cI{\mathcal{I}}
\def\Cset{\mathbb{C}}
\def\bm{m}
\def\bK{K}
\def\calN{\mathcal{N}}
%\def\1{\mathbbm{1}}
\def\EQM{\mathrm{EQM}}
\def\var{\mathrm{var}}

\def\hbbeta{\hat{\boldsymbol{\beta}}}

\def\umu{\underline{\mu}}
\def\omu{\overline{\mu}}
\def\unu{\underline{\nu}}
\def\onu{\overline{\nu}}
\def\Xset{\mathcal{X}}
\def\Yset{\mathcal{Y}}
\def\Zset{\mathcal{Z}}
\def\Xsigma{\mathcal{B}(\Xset)}
\newcommand{\pscal}[2]{\langle #1, #2 \rangle}
\def\lleb{\lambda^{\mathrm{Leb}}}

\def\intd{[}
\def\intg{]}
\def\PE{\esp}

\def\un{\1}
\def\bg{\mathbf{g}}
\def\bgam{\boldsymbol{\gamma}}
\def\bY{\mathbf{Y}}
\def\RR { {\mathbb{R}} }
\def\ZZ { {\mathbb{Z}} }
\def\NN { {\mathbb{N}} }
\newcommand{\bff}{\mathbf{f}}
\newcommand{\Y}[1]{\colorbox{yellow}{#1}}
\renewcommand{\hat}{\widehat}
\newcommand{\MISE}{{\mathrm{MISE}}}
\newcommand{\MSE}{{\mathrm{MSE}}}
\newcommand{\vect}{{\mathrm{vect}}}
\newcommand{\argmin}{\mathop{\mathrm{argmin}}}
\newcommand{\pen}{\mathop{\mathrm{PEN}}}
\newcommand{\CV}{\mathop{\mathrm{CV}}}
\newcommand{\tore}   { {\mathbb{T}} }
\newcommand{\supp}{\mathop{\mathrm{Supp}}}
\newcommand{\calW}{\mathcal{W}}
\newcommand{\calJ}{\mathcal{J}}
\newcommand{\ImPart}{{\mathrm{Im}}}
\newcommand{\RealPart}{{\mathrm{Re}}}
%% parametre d'int�ret 
\def\parami{g}
%%%%%%%%%%%%%%%%%%%%%%%%%%%%%%%%%%%%%%%%%%%%%%%%%%%%%%%%%%%%%%%%%%%%%%%%%
% general-def.tex       Definitions communes a tout le texte
%
%%%%%%%%%%%%%%%%%%%%%%%%%%%%%%%%%%%%%%%%%%%%%%%%%%%%%%%%%%%%%%%%%%%%%%%%%
%%%%%%%%%%% QUELQUES DEFINITIONS
\newcommand{\addots}
{
    \mkenrn2mu\raisebox{1pt}{.}
    \mkern2mu\raisebox{4pt}{.}
    \mkern2mu\raisebox{7pt}{.}
}
\def\rset{\mathbb{R}}
\def\cset{\mathbb{C}}
\def\zset{\mathbb{Z}}
\def\nset{\mathbb{N}}
\def\qset{\mathbb{Q}}
\def\Cset{\mathbb{C}}
\def\Zset{\mathbb{Z}}
% et d'autres
\newcommand{\signe}{\mathrm{sgn}}
\newcommand{\rect}{\mathrm{rect}}
\newcommand{\sinc}{\mathrm{sinc}}
\newcommand{\var}{\mathrm{var}}
\newcommand{\cov}{\mathrm{cov}}
\newcommand{\vp}{{\rm vp}}
\newcommand{\erf}{{\rm erf}}
\def\mod{{\ \rm mod\ }}
% Environnements
% Environnement encadre pour les modeles
\newenvironment{encadr}%
 {\begin{Sbox}\begin{minipage}}%
 {\end{minipage}\end{Sbox}\fbox{\TheSbox}}
% Listes pour les hypotheses dans les theoremes
\newenvironment{mathlist}%
  {\renewcommand{\labelenumi}{\textnormal{(\arabic{enumi})}}\begin{enumerate}}%
  {\end{enumerate}}
\newcommand{\ind}[2]{ \index{#1 \subitem{#2}} }

% Lettres r�serv�es � des nombres particuliers
\def\rmi{\mathrm{i}}
\def\rme{\mathrm{e}}
\def\rmd{\mathrm{d}}
% Abr�viations (attention : ne doivent pas figurer dans les maths)
\newcommand{\etal}{{\em et al.}}
\newcommand{\ie}{{\em i.e.}}
\newcommand{\NB}{\emph{N.B}}
\newcommand{\ssi}{si et seulement si}
% Non utilisees pour l'instant
\newcommand{\cad}{\hbox{c.-�-d.}}
\newcommand{\cf}{cf.}
\newcommand{\etc}{etc.}
\newcommand{\pex}{p.ex.}
\newcommand{\ps}{p.s.}
\newcommand{\va}{v.a.}
\newcommand{\ea}{e.a.}
\newcommand{\iid}{i.i.d.}
\newcommand{\iidsim}{\overset{\mathrm{i.i.d.}}{\sim}}
\newcommand{\resp}{resp.}
% Alphabet blackboard
\newcommand{\Un}{{\mathrm{1 \mkern-4.2mu I}}}
\newcommand{\eqdef}{\ensuremath{\stackrel{\mathrm{def}}{=}}}
\newcommand{\eqd}{\overset{\text{loi}}{=}}
\newcommand{\ud}{\mathrm{d}} 
\newcommand{\point}{\,\cdot\,}
\newcommand{\bpoint}{\,\boldsymbol{\cdot}\,}
% Probabilites, moments convergence, lois
\def\prob{\mathbb{P}}
\def\esp{\mathbb{E}}
\def\PP{\prob}
\def\PE{\esp}
\def\EE{\PE}
\newcommand{\hEs}{(\Omega,{\cal A}, \prob)}
\newcommand{\Pth}{{\loi_{\theta}}}
\newcommand{\cond}{\vert}
\newcommand{\Ptha}{\loi_{\theta \cond A_{i}}}
\newcommand{\Eth}{{\esp_{\theta}}}
\newcommand{\pth}{{p_{\theta}}}

\def\Var{\mathbb{ V}\mathrm{ar}}
\def\Cov{\mathbb{C}\mathrm{ov}}
\def\Covth{\Cov_{\theta}}
\def\Varth{\Var_{\theta}}

% symboles de loi
\def\loi{\mathrm{P}}
\def\loiemp{\mathrm{P_n}}
\def\loiunif{\mathcal{U}\mathit{nif}}
\def\loibeta{\mathcal{B}\mathit{eta}}
\def\loimulti{\mathcal{M}\mathit{ulti}}
\def\loiber{\mathcal{B}\mathit{er}}
\def\loibin{\mathcal{B}\mathit{in}}
\def\loihyper{\mathcal{H}\mathit{yper}}
\def\loigauss{\mathcal{N}}
\def\loigamma{\mathcal{G}\mathit{amma}}
\def\loipoisson{\mathcal{P}\mathit{oiss}}
\def\loistudent{\mathbf{T}}
\def\loifisher{\mathbf{F}}
\def\loikhi2{\mathbf{\chi^2}}
\def\loiexp{\mathcal{E}}

% Convergence en loi
\newcommand{\cl}{{\:\stackrel{\calL}{\longrightarrow}\:}}
\newcommand{\cla}[1]{{\:\stackrel{\calL_{#1}}{\longrightarrow}\:}}
\newcommand{\clt}{\cla{\theta}}
% Convergence en probabilit�
\newcommand{\cp}{{\:\stackrel{\prob}{\longrightarrow}\:}}
\newcommand{\cpa}[1]{{\:\stackrel{\prob_{#1}}{\longrightarrow}\:}}
\newcommand{\cpt}{\cpa{\theta}}
\newcommand{\cps}{{\:\stackrel{\prob-\mathrm{p.s.}}{\longrightarrow}\:}}
\newcommand{\cpsa}[1]{{\:\stackrel{\prob_{#1}-\mathrm{p.s.}}{\longrightarrow}\:}}
% Identite des distributions
\newcommand{\dist}{\sim}
% Convergence �troite
\newcommand{\ce}{\leadsto}
% Asymptotiquement Gaussien
\newcommand{\isan}[2]{\ \equiv \ {\cal AN}\left (#1,#2 \right )\ }
% Experience statistique
\newcommand{\expe}{(\Xset, \, \mathcal{B}(\Xset),\, (\Pth \,;\, \theta \in \Theta \subset \rset^k))}
\newcommand{\rexpe}{(\Xset, \, \mathcal{B}(\Xset),\, (\Pth \,;\, \theta \in \Theta \subset \rset))}
% Divers
\newcommand{\Tr}{{\mathrm{Tr}}}
\newcommand{\diag}{{\mathrm{diag}}}
\def\vol{{\rm vol}}
\newcommand{\fgauss}{\frac{1}{( 2 \pi \sigma^2 )^{^{\frac{n}{2}}}}}
\newcommand{\range}{{\rm Im}}
\newcommand{\rang}{{\rm rang}}
\def\compose{\circ}
% Dans moments.tex
\newcommand{\ketchit}{{o_{P_\theta}}}
\def\mutu{\mu_1^{(t)}}
\def\mutd{\mu_2^{(t)}}
\def\mt{m^{(t)}}
\def\hmt{\hat{m}_n^{(t)}}
\def\gt{g^{(t)}}
\def\htg{\hat{\theta}_n^\Gamma}
% Adherence
\def\adh{{\rm adh}}
% Differentiation
\newcommand{\Df}[3]{{\nabla_{#1}#2(#3)}}
\newcommand{\DDf}[3]{{\nabla^2_{#1}#2(#3)}}
\newcommand{\CDf}[4]{ {\frac{ \partial #3}{\partial #1_{#2}}(#4)}}
\newcommand{\CDfs}[3]{{\frac{ \partial #2}{\partial #1}(#3)}}
%%%%%%%%%%%%%%%%%%%%%%%%%%%
%% qq lettres en gras et/ou avec des chapeaux
\newcommand{\bI}{{\mathbf I}}
\newcommand{\bzero}{{\mathbf 0}}
\newcommand{\bun}{{\mathbf 1}}
\newcommand{\1}{\mathbbm{1}}
\newcommand{\mb}{\mathbf}
%
% lettres en GRAS
\def\bA{\mathbf{A}}
\def\ba{\mathbf{a}}
\def\bb{\mathbf{b}}
\def\bB{\mathbf{B}}
\def\bC{\mathbf{C}}
\def\bc{\mathbf{c}}
\def\bD{\mathbf{D}}
\def\bd{\mathbf{d}}
\def\bE{\mathbf{E}}
\def\be{\mathbf{e}}
\def\grasf{\mathbf{f}}
\def\bF{\mathbf{F}}
\def\bg{\mathbf{g}}
\def\bG{\mathbf{G}}
\def\bh{\mathbf{h}}
\def\bH{\mathbf{H}}
\def\bJ{\mathbf{J}}
\def\bk{\mathbf{k}}
\def\bK{\mathbf{K}}
\def\bL{\mathbf{L}}
\def\bM{\mathbf{M}}
\def\bn{\mathbf{n}}
\def\bO{\mathbf{O}}
\def\bp{\mathbf{p}}
\def\bP{\mathbf{P}}
\def\bQ{\mathbf{Q}}
\def\bR{\mathbf{R}}
\def\br{\mathbf{r}}
\def\bs{\mathbf{s}}
\def\bS{\mathbf{S}}
\def\bt{\mathbf{t}}
\def\bT{\mathbf{T}}
\def\bu{\mathbf{u}}
\def\bU{\mathbf{U}}
\def\bN{\mathbf{N}}
\def\bv{\mathbf{v}}
\def\bV{\mathbf{V}}
\def\bw{\mathbf{w}}
\def\bW{\mathbf{W}}
\def\bx{\mathbf{x}}
\def\bX{\mathbf{X}}
\def\by{\mathbf{y}}
\def\bY{\mathbf{Y}}
\def\bz{\mathbf{z}}
\def\bZ{\mathbf{Z}}
%
\def\eqsp{\;}

%
% lettres grecques en gras
\def\bGamma{\boldsymbol{\Gamma}}
\def\balpha{\boldsymbol{\alpha}}
\def\bbeta{\boldsymbol{\beta}}
\def\bepsilon{\boldsymbol{\epsilon}}
\def\bphi{\boldsymbol{\phi}}
\def\bnu{\boldsymbol{\nu}}
\def\bmu{\boldsymbol{\mu}}
\def\boeta{\boldsymbol{\eta}}
\def\bgamma{\boldsymbol{\gamma}}
\def\btheta{\boldsymbol{\theta}}
\newcommand{\mbf}[1]{\mbox{\boldmath$#1$}}
% lettres "CHAPEAU"
\newcommand\ha{{\hat a}}
\newcommand{\TnG}{{{\theta}_n^{G}}}
\newcommand\hf{{\hat f}}
\newcommand{\hg}{\hat{g}}
\newcommand{\hr}{{\hat r}}
\newcommand{\hR}{{\hat R}}
\newcommand\hs{{\hat s}}
\newcommand{\hS}{{\hat S}}
\newcommand{\hT}{{\hat T}}
%\newcommand{\hsigmad}{{{\hat \sigma}^2}}
\newcommand{\hbS}{{\mathbf {\hat S}}}
\newcommand{\hbx}{{\mathbf {\hat x}}}
\def\halpha{{\hat \alpha}}
\def\hbeta{{\hat \beta}}
\def\hph{{\hat{\phi}}}
\def\hbph{{\hat{\bphi}}}
\def\hth{{\hat{\theta}}}
\def\hbth{{\hat{\btheta}}}
\def\thj{{\theta^{(j)}}}
\def\thju{{\theta^{(j+1)}}}
% lettres tilde
\newcommand{\tg}{\tilde{g}}
\newcommand{\tth}{{\tilde \theta}}
% Lettres caligraphiques
\def\calL{\mathcal{L}}
\def\calA{\mathcal{A}}
\def\calM{\mathcal{M}}
\def\calI{\mathcal{I}}
\def\calC{\mathcal{C}}
\def\calF{\mathcal{F}}
\def\calS{\mathcal{S}}
\def\calE{\mathcal{E}}
\def\calU{\mathcal{U}}
\def\calX{\mathcal{X}}
\def\calY{\mathcal{Y}}
\def\calZ{\mathcal{Z}}
\def\calP{\mathcal{P}}
\def\calB{\mathcal{B}}
\def\calK{\mathcal{K}}
\def\calR{\mathcal{R}}
\def\calT{\mathcal{T}}
\def\calG{\mathcal{G}}
\def\calM{\mathcal{M}}
\def\cF{\mathcal{F}}
\def\cE{\mathcal{E}}
\def\cB{\mathcal{B}}
\def\limn{\lim_{n \rightarrow \infty}}
\def\cH{\mathcal{H}}
\def\cG{\mathcal{G}}
\def\cI{\mathcal{I}}
\def\Cset{\mathbb{C}}
\def\bm{m}
\def\bK{K}
\def\calN{\mathcal{N}}
%\def\1{\mathbbm{1}}
\def\EQM{\mathrm{EQM}}
\def\var{\mathrm{var}}

\def\hbbeta{\hat{\boldsymbol{\beta}}}

\def\umu{\underline{\mu}}
\def\omu{\overline{\mu}}
\def\unu{\underline{\nu}}
\def\onu{\overline{\nu}}
\def\Xset{\mathcal{X}}
\def\Yset{\mathcal{Y}}
\def\Zset{\mathcal{Z}}
\def\Xsigma{\mathcal{B}(\Xset)}
\newcommand{\pscal}[2]{\langle #1, #2 \rangle}
\def\lleb{\lambda^{\mathrm{Leb}}}

\def\intd{[}
\def\intg{]}
\def\PE{\esp}

\def\un{\1}
\def\bg{\mathbf{g}}
\def\bgam{\boldsymbol{\gamma}}
\def\bY{\mathbf{Y}}
\def\RR { {\mathbb{R}} }
\def\ZZ { {\mathbb{Z}} }
\def\NN { {\mathbb{N}} }
\newcommand{\bff}{\mathbf{f}}
\newcommand{\Y}[1]{\colorbox{yellow}{#1}}
\renewcommand{\hat}{\widehat}
\newcommand{\MISE}{{\mathrm{MISE}}}
\newcommand{\MSE}{{\mathrm{MSE}}}
\newcommand{\vect}{{\mathrm{vect}}}
\newcommand{\argmin}{\mathop{\mathrm{argmin}}}
\newcommand{\pen}{\mathop{\mathrm{PEN}}}
\newcommand{\CV}{\mathop{\mathrm{CV}}}
\newcommand{\tore}   { {\mathbb{T}} }
\newcommand{\supp}{\mathop{\mathrm{Supp}}}
\newcommand{\calW}{\mathcal{W}}
\newcommand{\calJ}{\mathcal{J}}
\newcommand{\ImPart}{{\mathrm{Im}}}
\newcommand{\RealPart}{{\mathrm{Re}}}
%% parametre d'int�ret 
\def\parami{g}
\newcommand{\blambda}{\boldsymbol{\lambda}}


% \def\exo#1 {\vskip 8mm {\bf Exercice #1 $-$ }}
% \def\qa {{\bf 1) }}
% \def\qb {{\bf 2) }}
% \def\qc {{\bf 3) }}
% \def\qd {{\bf 4) }}
% \def\qe {{\bf 5) }}
% \def\qf {{\bf 6) }}
% \def\qg {{\bf 7) }}

% \def \1{\mathbbm{1}}
% \def\rme{\textrm{e}}

\begin{document}
\Opensolutionfile{ans_file}[Ans-miniprojet]


\centerline{\Large\textbf{Mini-Projet}}
\ \\


Pour la reproductibilité des  questions numériques, il est conseillé de fixer la \og
graine\fg{} du générateur de nombres pseudo-aléatoires, en haut de
votre script,  en utilisant la
fonction \texttt{set.seed} de \lstinline+R+, par exemple : 
\begin{lstlisting}
 set.seed(42,kind="Marsaglia-Multicarry")
\end{lstlisting}

On rappelle les résultats suivants \\~\\
\fbox{
  \begin{minipage}{1.0\linewidth}
    \begin{enumerate}
    \item {\bf loi des grands nombres } \\
Soit $Z:\Omega\to\rset$  une variable aléatoire sur un espace
probabilisé $(\Omega,\calF,\PP)$ telle que
$\esp(|Z|)<+\infty$, et soit $(Z_i)_{i\ge 0}$ est un
échantillon $\iid$ de même loi que $Z$, défini sur le même espace.     Il existe $N\subset\Omega$ tel que $\PP(N) = 0$ et 
\[
\forall \omega\in \Omega\setminus N,\qquad  \frac{1}{n} \sum_{i=1}^n Z_i(\omega)
\xrightarrow[n\to\infty]{} \esp(Z).
\]  
Autrement dit, la moyenne empirique des $Z_i$ converge $\PP$-presque
sûrement  vers $\esp(Z)$.
\item {\bf Loi du $\chi^2$ (`Chi 2')} \\
  Si $Y_1,\ldots,Y_n$ sont des variables aléatoires i.i.d.~de loi normale centrée réduite, de moyenne empirique $\bar Y$, alors la variable aléatoire 
$V = (Y_1 - \bar Y)^2+\ldots + (Y_n - \bar Y)^2$
suit une loi du $\chi^2$ à $n-1$ degrés de libertés.

\item {\bf Loi Gamma} Une variable aléatoire $Y$ suit une loi Gamma de
  paramètres $a$ et $b $  ($a>0$ et $b >0$), notée
  $\loigamma(a,b )$,  si
  elle admet une densité  par rapport à la mesure
de Lebesgue  donnée par 
\[
f_{(a,b )}^{\cal G}(y) = \un_{y>0}\frac{b^a}{\Gamma(a)}
y^{a -1} e^{-b y}. 
\]
On rappelle que pour $a > 0$, $\Gamma(a+1) =
a\Gamma(a).$ Si $Y\sim\loigamma(a, b)$, on a 
\[
\esp_{a,b}(Y) = \frac{a}{b}\quad;\quad
\Var_{a,b}(Y) = \frac{a}{b^2}.
\]

    \end{enumerate}
   \end{minipage}
}\ \\

On s'intéresse à la distribution de la taille des fichiers stockés dans un répertoire. Le jeu de données se trouve ici :
\href{http://perso.telecom-paristech.fr/~bonald/filesize.txt}{http://perso.telecom-paristech.fr/$\sim$bonald/filesize.txt}

Ce jeu de données  comporte la taille en octets de $n = 400$ fichiers, soit $x = (x_1,\ldots,x_n)$.

~\\
\NB\ Les quantiles de la loi log-normale sont disponibles numériquement dans \texttt{R}, tout comme ceux de la loi normale, grâce aux fonctions \lstinline{qnorm} et \lstinline{qlnorm}

\begin{exercice}[Analyse exploratoire (2pts)]~

  \begin{enumerate}
    \item Tracer un histogramme de la loi empirique de la taille des fichiers en échelle logarithmique (soit $\log(x_1),\ldots,\log(x_n)$).
  \item superposer l'histogramme (avec l'option \lstinline+probability = TRUE+) et la densité d'une loi normale de moyenne et variance respectivement égales à la moyenne et la variance empiriques des $\log(x_i)$. 

\end{enumerate}  
\end{exercice}

Au vu de l'exercice 1,
On modélise ces données comme des échantillons i.i.d.~d'une loi log-normale de paramètres $\mu,\sigma^2$ (la taille  de chaque fichier est donc représentée par une variable aléatoire $X$   telle que $\log(X)$ suit une loi normale d'espérance $\mu$ et de variance $\sigma^2$). On note $\theta = (\mu,\sigma^2)$.
Certaines questions font appel à la loi du $\chi^2$. 
\begin{exercice}[Estimation ponctuelle (7 pts)]~

  \begin{enumerate}
\item Calculer la densité par rapport à la mesure de Lebesgue de la loi log-normale de paramètre $\theta=(\mu,\sigma^2)$, en utilisant un changement de variables approprié.
  
\item Calculer l'estimateur du maximum de vraisemblance $\hat \theta = (\hat \mu,\hat{\sigma^2})$ de $\theta$. Cet estimateur est-il biaisé ? Si oui, ce biais est-il significatif pour ce jeu de données ?

\item Représenter la loi associée pour le jeu de données considéré sur le même graphique que la loi empirique (toujours en échelle logarithmique).

\item Calculer le risque quadratique associé à l'estimateur $\hat \mu$ de $\mu$. Cet estimateur est-il efficace ?

\item On s'intéresse maintenant à la taille moyenne des fichiers, $g(\theta) = {\rm E}_\theta(X)$. L'estimateur $g(\hat \theta)$ est-il efficace ? Comparer la valeur obtenue pour ce jeu de données avec la moyenne empirique.
\item Enfin, on s'intéresse au quantile de niveau $0.95$, soit la valeur $q(\theta)$ tel que  ${\rm P}_\theta(X\le q(\theta)) = 0.95$. On cherche à estimer $\log q(\theta)$. L'estimateur $\log q(\hat \theta)$ est-il efficace ? Comparer la valeur obtenue pour ce jeu de données avec le quantile équivalent   de la loi empirique.

\end{enumerate}

\end{exercice}

\begin{exercice}[Taille de fichiers et modélisation Bayésienne (7 pts)]
  On considère le même jeu de données qu'à l'exercice précédent et on s'intéresse au paramètre $\sigma^2$ de la loi log-normale de la taille des fichiers. On considérera dans toute la suite que le paramètre $\mu$ est connu, on prendra $\mu = 9.1$ dans les questions numériques.
  \begin{enumerate}
  \item Justifier l'hypothèse `$\mu$ connu, $\mu = 9.1$': pour cela, estimer l'écart type de l'estimateur du maximum de vraisemblance pour $\mu$ et comparer à une grandeur de référence qui vous parait pertinente. % avec  l'écart-type estimé  de $\log(X)$.
  \end{enumerate}
   On se place désormais dans un cadre bayésien  pour l'estimation de $\sigma^2$.
   Pour des raisons pratiques qui apparaîtront ci-dessous, on préfère travailler avec l'inverse de $\sigma^2$, $\lambda = 1/\sigma^2$. On choisit comme prior sur $\lambda$ une loi Gamma $\pi = \loigamma(a,b)$ avec $a>0, b>0$ des hyper-paramètres fixés par le statisticien. 
\begin{enumerate}[resume]%%\setcounter{enumi}{1}
\item En l'absence d'information pertinente a priori sur la taille des
  fichiers, on choisit un prior `large'. Déterminer $a,b$ pour que
  $\esp_{\pi}[\blambda] = 1$ et $\Var_\pi(\blambda) = 10$.
\item Déterminer l'expression de la loi a posteriori de $\blambda$ pour $n$ données
  $(x_1,\ldots,x_n)$.  Calculer numériquement  les paramètres
  de cette loi a posteriori pour le jeu de données fourni.
\item En déduire l'expression de l'estimateur de l'espérance a
  posteriori pour le paramètre $\blambda$. Comparer avec le résultat
  obtenu par maximum de vraisemblance.

\item Tracer sur un même graphique la densité de la loi a priori entre $0$ et $1$, celle de la loi a posteriori. Indiquer par des lignes verticales l'estimateur de l'espérance a posteriori et $1/\hat{\sigma^2}$.
  
\end{enumerate}
On veut construire l'espérance a posteriori $ \hat h  = \esp_{\pi}[h(\blambda) \,|\, x_1,\ldots, x_n]$ de la quantité d'intérêt
  $h(\lambda) = \log q_\lambda(0.95)$ avec $q_\lambda$ le quantile de la loi log-normale de paramètres $(\mu=9.1, \sigma^2 = 1/\lambda)$. On ne dispose   pas d'expression explicite pour $h(\lambda)$ ni pour $\hat h$. Cependant, comme précisé en introduction du projet, les quantiles de la loi log-normale et de la loi normale sont disponibles numériquement dans \texttt{R}. % , tout comme ceux de la loi normale, grâce aux fonctions \lstinline{qnorm} et \lstinline{qlnorm}
\begin{enumerate}[resume]%%\setcounter{enumi}{5}
\item 
  Simuler un échantillon $(\lambda_i)_{i = 1,\ldots, M}$
  indépendant et identiquement distribué selon la loi a posteriori,
  avec $M$ suffisamment grand, de manière à approcher $\hat h$ par une
  moyenne empirique $\tilde h = \frac{1}{M} \sum_{i=1}^M Z_i$ avec $Z_i$ convenablement construit à partir de $\lambda_i$ et d'une fonction quantile:
  \begin{enumerate}
  \item Expliciter $Z_i$ et fournir l'estimation $\tilde h$ demandée.
\item  Donner une estimation de l'écart-type
  de $\tilde h$, conditionnellement à $x_1, \ldots, x_n$.
\item Tracer sur le même graphique, en fonction de $M$, l'évolution de $\tilde h$ et
  d'un encadrement de $\tilde h$ de largeur 2 écarts-types, pour l'écart-type calculé ci-dessus. 
  \end{enumerate}
  
\end{enumerate} 
\end{exercice} 



\begin{exercice}[Test d'hypothèses (4 pts)]
  L'administrateur du réseau cherche à tester l'hypothèse
  $H_0: \sigma^2 \le 8$ contre $H_1: \sigma^2>8$. 
  \begin{enumerate}
  \item Construire un  test de niveau $\alpha=0.05$ de $H_0$ contre $H_1$ basé sur la statistique
    \[
      \varphi(X_1,\ldots, X_n)  = \sum_{i=1}^n (\log(X_i) - \overline{\log(X)})^2
    \]
    avec $\overline{\log(X)} = \frac{1}{n}\sum_{i=1}^n \log(X_i)$.  Préciser la région d'acceptation en fonction des quantiles d'une loi que l'on précisera. 
  \item Quel est le résultat du test sur le jeu de données considéré ?
    \item Quel est le seuil minimal $\sigma_0$ tel que l'hypothèse $\tilde H_0: \sigma \le \sigma_0$ soit rejetée par un test de niveau $0.05$ ?
  \end{enumerate}
\end{exercice}

\end{document} 








%%% Local Variables:
%%% mode: latex
%%% TeX-master: t
%%% End:
